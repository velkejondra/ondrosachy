\documentclass[a4paper, 12pt]{report}
\usepackage{monapack}
\usepackage{hyperref}
\usepackage{amsmath}

\usepackage[outputdir=../]{minted}
\usepackage{xcolor}

\student{Ondřej Polanecký}
\trida{B4.I}
\obor{programator}
\bydliste{L. Janáčka 1266}
\datumNarozeni{3.1.2002}
\vedouci{Mgr. Milan Průdek}
\nazevPrace{Šachový bot}
\cisloPrace{23}
\skolniRok{2020/2021}
\reditel{Ing. Jiří Uhlík}

\zacatek

\titulniStrana


\anotace
Tato maturitní práce představuje můj šachový engine a vysvětluje některé techniky, které šachoví
boti využívají.
Také obsahuje GUI s návodem na používání, aby si každý mohl snadno vyzkoušet zahrát si proti šachovému botu.
\newline\textbf{Klíčová slova:} Šachy, Šachový bot
\annotation
This graduation work presents my chess engine and explains some techniques used by chess engines.
It aslo contains GUI with guide how to use it, so everyone can easily play against my chess bot.
\newline\textbf{Keywords:} Chess, Chess bot, Chess engine
\podekovani
Děkuju pani Maříkový.
\obsah


\chapter{Úvod}
Tato maturitní práce představuje můj šachový engine a popisuje různé techniky, či algoritmy používané v šachových enginech.

V této práci nebudu popisovat pravidla šachů, budu počítat s vaší znalostí pravidel.


\chapter{Popis a struktura šachového enginu}


\section{Teorie her}


Šachy jsou stálé nevyřešená hra tzn. neví se jak vypadá bezchybná hra.
Jsou nevyřešené, protože počet možných pozic je enormní.
Už jenom po třech pohybech je 8902 možných variant hry.
Pro představu jsem vygeneroval graf pomocí programu Graphviz, který ukazuje jak se hra po třech tazích větví.
Viz obr.~\ref{fig:strom_tahu}
\begin{figure}
    \centering
    \includegraphics[scale=0.3]{images/graph.resized.jpg}
    \caption{větvění šachů po třech pohyběch}
    \label{fig:strom_tahu}
\end{figure}


Průměrný počet pohybů ve všech situací je přibližně 35,
průměrná hra má 80 pohybů, takže když chceme dostat hodně nepřesný odhad možných šachovým partií, tak nám stačí tyto dvě hodnoty umocnit
$35^{80} $\approx$ 10^{123} $\cite{shannon_number}.
Z tohoto enormního čísla vyplývá, že šachy nelze vyřešit hrubou silou a pravděpodobně šachy v nejbližší době, či dokonce nikdy nevyřešíme.
Pro zajímavost dáma má přibližně $5*10^{20}$ variací her a byla vyřešena v roce 2007.\cite{dama} Při perfektním zahrání od obou hráčů skončí hra remízou.\newline
Šachy jsou:
\begin{itemize}
    \item  hra s úplnými informacemi, tzn.
    oba hráči ví o všech informacích ve hře (vidí všechny figurky).
    Na druhou stranu u her s neúplnými informacemi (např. poker),
    všechny informace neznáme a engine by musel počítat s pravděpodobnostmi pro určité informace a na základě těchto pravděpodobností se rozhodovat.
    \item hra s nulovým součtem, tzn. jakoukoliv výhodu hráč získá na úkor protihráče.
\end{itemize}
\begin{flushleft}
    Díky tomu můžeme tvořit herní strom všech možných tahů do určité hloubky a relativně dobře odhadovat hodnoty pohybů.
    Herní strom vypadá přibližně takto.
    Viz obr.~\ref{fig:game_tree}
    \begin{figure}
        \centering
        \includegraphics[scale=0.2]{images/game_tree}
        \caption{herní strom piškvorek~\cite{game_tree}}
        \label{fig:game_tree}
    \end{figure}

\end{flushleft}


\section{Dělení šachových enginů}
V dnešní době existují v podstatě dva druhy šachových botů.
\begin{itemize}
    \item Šachový bot založený na alfa-beta vyhledávání
    \begin{itemize}
        \item Šachový engine prochází rekurzivně herní strom do určité hloubky a na konečných uzlech stromu zhodnotí pozici.
        \item Hodnocení pozic je děláno funkcí vytvořenou programátorem a nedostatky evaluační funkce jsou dohnány prohledáváním spousty pozic do velké hloubky.
        \item Detaily budou v kapitole~\ref{sec:alpha–beta-pruning}
    \end{itemize}
    \item Šachový bot založený na neuronových sítí a prohledávání stromu metodou Monte Carlo
    \begin{itemize}
        \item Šachový engine prochází rekurzivně herní strom do určité hloubky a na konečných uzlech stromu zhodnotí pozici.
        \item Hodnocení pozic je děláno neuronovou sítí, která byla natrénována na hraním proti sobě.
        \item Na vyhledávání se používá \href{https://en.wikipedia.org/wiki/Monte_Carlo_tree_search}{Monte-Carlo tree search}.
        Oproti Alfa-Beta vyhledávání prohledává podstatně méně pozic, ale prohledává pouze pozice s velkou šancí na úspěšnost.
    \end{itemize}
\end{itemize}


\section{Struktura šachového enginu}\label{sec:struktura-šachového-enginu}
Můj šachový engine bude založen na alfa-beta vyhledávání, takže budu popisovat hlavně tento typ enginů.

Šachový engin by měl mít:
\begin{itemize}
    \item Generátor legálních pohybů
    \begin{itemize}
        \item Třída má na starosti generování legálních pohybů v dané situaci
        \item Většina šachových enginů nejdřív vygeneruje všechny pseudo-legální pohyby\footnote{Pohyby, které odpovídají tomu, jak se mají figurky hýbat, ale je u nich možnost, že by dostali svého krále do šachu.},
        otestuje pohyby a vyřadí u kterých je vlastní král v šachu.
    \end{itemize}
    \item Třída na reprezentaci stavu hry
    \begin{itemize}
        \item Udržuje informace o pozicích, právech figurek a případně s nimi hýbe
        \item na udržování pozic se většinou použivají bitboardy\footnote{Bitboard nebo bitmapa je 64 bitové číslo kde v jeho binární podobně každá jednička znamená zabranou pozici na šachovém poli.}.
        Detaily o bitboardech v kapitole~\ref{sec:bitboard_kapitola}
    \end{itemize}
    \item Transpoziční tabulka
    \begin{itemize}
        \item Hašovací tabulka která obsahuje hashe pozic a nejlepší pohyb pro určitou pozici.
        \item Může obsahovat již prohledané pozice nebo pozice získané z předem vytvořené databáze pohybů.
        \item Detaily v kapitole~\ref{trans_tabulka}
    \end{itemize}
    \item Evaluace pozic
    \begin{itemize}
        \item Hodnotí pozici podle hodnoty, pozice a struktury figurek.
        \item Klíčová pro chování šachového enginu.
        \item Na základě ní se rozhoduje v prohledávání pozic.
    \end{itemize}
    \item Prohledávací
\end{itemize}


\chapter{Implementace}\label{ch:implementace}


\section{Bitboardy}
\label{sec:bitboard_kapitola}

Bitboard je 64 bitové číslo.
Každá pozice bitu koresponduje s pozicí na herní ploše.
Pokud je pozice na herní ploše zabrána, tak je korespondující bit v bitboardu nastaven na 1.
Aby se dali rozlišit jednotlivé figurky, je třeba udržovat v paměti bitboard pro každý druh a barvu figurek.
Viz. ilustrační obrázek \ref{fig:bitboard}
Důvod proč se využívají bitboardy je kvůli rychlosti generování pohybů.
Například u generování pohybů pro koně jsme schopni si pro každou pozici předpočítat možné pohyby koně.
Také můžeme na bitboardy používat logické operace, takže můžeme například všechny pěšáky posunout o 8 bitů doprava (pohyb nahoru) v jednom clock cyclu.
V c++ jsem použil datovou strukturu unsigned long long.

\begin{figure}
    \centering
    \includegraphics[scale=0.7]{images/bitboard}
    \caption{Reprezentace herní plochy pomocí bitboardů\cite{bitboards_wiki}}
    \label{fig:bitboard}
\end{figure}

\newpage

V kódu udržuji bitboardy ve třídě Board.
Navíc k tomu mám metody, které určité bitboardy spojí logickou operací OR.
Takhle vypadá definice třídy.

\begin{minted}[
    frame=lines,
    framesep=2mm,
    baselinestretch=1.2,
    fontsize=\footnotesize,
    linenos
]
{c++}
typedef unsigned long long bitboard;
class Board {
    public:
    bitboard all_bitboards[2][6]{};

    // spojene pozice

    // vraci bitboard figurek urcite barvy
    bitboard PiecesOfColor(bool color);

    // vraci bitboard vsech figurek
    bitboard AllPieces();
}
\end{minted}


\section{Generování pohybů}\label{sec:generování-pohybů}
Generování pohybů při používání bitboardů se dělá specifickým způsobem.
Pro figurky, které se hýbou nezávisle na tom kde jsou postaveny ostatní figurky, je to lehké.
Pouze pro všech 64 pozic kde se může figurka nacházet si předpočítáme možné pozice kam může jít.
Všechny tyto pozice uložíme do pole a jako index použijeme pozici bitu.
Když potom narazíme na figurku na určitém poli, stačí vzít předpočítaný bitboard.
Generování potom probíhá postupným procházení předpočítaného bitboardu.
Aby bylo procházení bitboardu co nejrychlejší používám funkci \_\_builtin\_ffsll().
Tato funkce je závislá na kompilátoru (GCC) a snaží se použít přímo assembly instrukci, pokud je dostupná.
Funkce vrátí velmi rychle index nejméně významného bitu, který je jedna.
Tím dostanu index pozice kam může figurka jít.
Prohledaný bit poté nastavím na nulu abych přístě dostal další jedničku.

Na ukázku přikládám úryvek kódu pro generování pohybů koňem.

\begin{minted}[
    frame=lines,
    framesep=2mm,
    baselinestretch=1.2,
    fontsize=\footnotesize,
    linenos
]
{c++}
    bitboard enemy_pieces = board.EnemyPieces();
    bitboard my_pieces = board.MyPieces();
    bitboard my_knights = board.all_bitboards[board.on_turn][KNIGHT];
    int from;
    int to;
    while (my_knights) {
        // nacteni pozice dalsiho kone
        from = __builtin_ffsll(my_knights);
        from--;
        // smazani nacteneho kone
        my_knights ^= 1ULL << (from);
        // utoky ze soucasneho kone
        bitboard attacks = precomp.precomputed_knights[from];
        // odstraneni pohybu, ktere by skoncili na nejake me figurce
        attacks &= ~my_pieces;
        while (attacks) {
            //nacteni pozice pro pohyb
            to = __builtin_ffsll(attacks);
            to--;

            // smazani nactene pozice
            attacks ^= 1ULL << (to);
            if (enemy_pieces >> to & 1ULL) {
                // pohyb sebere figurku
                moves.push_back(Move{from, to, KNIGHT, board.getPieceAt(to), true});
            }
            else {
                // pohyb nesebere figurku
                moves.push_back(Move{from, to, KNIGHT});
            }
        }
\end{minted}


Vetší problém je si předpočítat pohyby pro figurky u kterých záleží na jakých pozicích jsou ostatní figurky (věž, střelec, dáma).
Je snadné
Když si vezmeme třeba tento příklad.\ref{fig:precomputed_rook1}
\begin{figure}
    \centering
    \includegraphics[scale=0.4]{images/precomputed_rook1}
    \caption{Možnosti pohybu věže}
    \label{fig:precomputed_rook1}
\end{figure}

Vidíme, že nás tedy zajímá jaké figurky blokují pohyb věže.
Můžeme si teda spočítat všechny možnosti kde můžou být.
Zajímají nás jenom figurky, které jsou na
V tomto konkrétním případě je to 14 pozic kde můžou být figurky umístěny.
To je $2^{14}=16384$ pozic jak můžou být figurky uspořádány.
Dokonce to číslo můžeme ještě zmenšit, protože na úplně krajních pozicích nezáleží.
Krajní pozice totiž nic neblokují.
Takže v tomto případě to je $2^{10} = 1024$ pozic.
Což v dnešní době není problém v paměti udržet.
Teď ale musíme vyřešit jak budeme pozice v poli indexovat.
Jedna možnost je použít hashmapu a jako index použít bitboard blokujících figurek.
Je tu ale rychlejší a paměťově méně náročnější možnost.
Jsou to tzv.
magické bitboardy.
Jde o to, že v celém bitboardu nás zajímají pouze určité bity.
A my je můžeme rychlou metodou vyjmout z bitboardu a indexovat pouze podle těchto určitých pozic.
Dělá se to tak, že se určitým číslem (magickým) vynásobí bitboard a to číslo je tak šikovné, že nám posune chtěné pozice na prvnich x bitů.
V minulém případě by to posunulo těch 10 bitů na začátek čísla.
Pak jenom ořízneme zbytek bitboardu a indexujeme podle těchto 10 bitů.
Tyto čísla se získávají náhodným zkoušením a hledá se takové aby to namapovalo námi chtěné bity na co nejméně bitů.
V ideálním případě na tolik bitů kolik bitů máme.
Já jsem si svoje čísla sám nepočítal.
Pouze jsem vzal čísla, které už někdo spočítal\cite{rustad-elliott}

Ukážu na příkladu jak by se vytvořil index pro tuto situaci.


\begin{minted}[
    frame=lines,
    framesep=2mm,
    baselinestretch=1.2,
    fontsize=\footnotesize,
]
{text}
bitboard bitů,                    bitboard obsazených                    bitboard
které nás zajímáají                      pozic                      blokujících figurek
. . . . . . . .                    1 1 . . . 1 . .                    . . . . . . . .
. . . 1 . . . .                    . 1 1 1 1 . . 1                    . . . 1 . . . .
. . . 1 . . . .                    . 1 . 1 . . . 1                    . . . 1 . . . .
. . . 1 . . . .                    . . . . . . . .                    . . . . . . . .
. 1 1 P 1 1 1 .         &          . 1 . . . . 1 .         =          . 1 . . . . 1 .
. . . 1 . . . .                    . . . 1 . . . .                    . . . 1 . . . .
. . . 1 . . . .                    . . . . . 1 . .                    . . . . . . . .
. . . . . . . .                    1 . . . . . . .                    . . . . . . . .

\end{minted}

Vezmeme bitboard bitů, které nás zajímají a provedeme bitový součin s bitboardem obsazených pozic.
Tímto dostaneme bitboard blokujících figurek, který už máme vyřešený v našem předpočítaném poli.
Jenom z něho musíme dostat co je to za index.

\begin{minted}[
    frame=lines,
    framesep=2mm,
    baselinestretch=1.2,
    fontsize=\footnotesize,
    escapeinside=||,
]
{text}
 . . . . . . . .                                   |\colorbox{green}{1 1 . 1 . . 1 1}|
 . . . 1 . . . .                                   |\colorbox{green}{. .}| . 1 . . . .
 . . . 1 . . . .       pole magických čísel        . . . . . . . .
 . . . . . . . .                                   . 1 . . . . . .
 . 1 . P . . 1 .   *      rook_magic[P]       =    . . . . 1 . . .
 . . . 1 . . . .                                   . . . . . . . .
 . . . . . . . .                                   . . 1 . . . . .
 . . . . . . . .                                   . . . . . . 1 .
\end{minted}

Vynásobením magickým číslem dostaneme index.
Zeleně zabarvená část vypočítaného čísla je index.
Teď už jenom stačí bitovým posunem dostat jenom posledních 10 bitů čísla a máme index.

\subsection{Debugování}\label{subsec:debugování}
Problém s debugováním generátoru pohybů je ohromné množství možných pozicí, takže je velmi obtížné hledat všechny chyby v generátoru pouhým náhodným hraním.
Proto byl vymyšlen perft (PERFormance Test)\cite{perft}.
Perft projíždí všechny možné pozice do určité hloubky a konečné uzly stromu sečte.
Potom výsledek porovnám se známými správnými výsledky.
\newline
Výsledek spočítáme klasickým DFS (Depht First Search).
Budeme rekurzivně volat naší funkci na všechny pohyby a až dorazíme do hloubky jedna tak vrátímě počet všech pozic.
Ty se potom v každé větvi sečtou.
Nakonec dostaneme všechny konečné pozice.
Tady je kód moji perft funkce.
\begin{minted}[
    frame=lines,
    framesep=2mm,
    baselinestretch=1.2,
    fontsize=\footnotesize,
    linenos
]
{c++}
unsigned long long perft(int depth, Board &board) {
    MoveGenerator gen = MoveGenerator();
    if (depth == 0) {
        return 1;
    }
    if (depth == 1) {
        return gen.getLegalMoves(board).size();
    }
    unsigned long long nodes = 0;
    for (auto move : gen.getLegalMoves(board)) {
        Board next_board = board;
        next_board.MakeMove(move);
        nodes += perft(depth - 1, next_board);
    }

    return nodes;
};
\end{minted}

Tuto funkci potom využiji na ověření správnosti generátoru takto.
Správné hodnoty konečných pozic jsem sebral z internetu.\cite{perft_results}
Poté jsem jenom pomocí c++ knihovny Catch2 porovnal hodnoty.
\begin{minted}[
    frame=lines,
    framesep=2mm,
    baselinestretch=1.2,
    fontsize=\footnotesize,
    linenos
]
{c++}
    REQUIRE(perft(1, board) == 20);
    REQUIRE(perft(2, board) == 400);
    REQUIRE(perft(3, board) == 8902);
    REQUIRE(perft(4, board) == 197281);
    REQUIRE(perft(5, board) == 4865609);
    REQUIRE(perft(6, board) == 119060324);
\end{minted}


\section{Prohledávání pozic}\label{sec:alpha–beta-pruning}
Alfa beta prunning je metoda na zvolení nejlepšího pohybu ve stromu pohybů a odřezávání nepotřebných větví.
Je to vylepšení algoritmu minimax.

\subsection{Minimax}
Minimax algoritmus rekurzivně projde herní strom do nějaké hloubky a na posledním uzlu vrátí evaluaci pozice.
Při vracení rekurze vždycky porovnám všechny potomky pozice a vyberu tu nejvýhodnější pro hráče právě na tahu.

Takhle vypadá vygenerovaný minimax strom, který ještě nemá ohodnoceny poslední uzly.~\ref{fig:minimax_tree}

\begin{figure}
    \centering
    \includegraphics[scale=0.22]{images/minimax_tree}
    \caption{Minimax strom. Barva uzlu určuje hráče na tahu v dané pozici}
    \label{fig:minimax_tree}
\end{figure}

Následně evaluační funkcí ohodnotím poslední uzly.\ref{fig:minimax_tree_last}
\begin{figure}
    \centering
    \includegraphics[scale=0.22]{images/minimax_tree_last}
    \caption{Minimax strom. Poslední pozice ohodnocené. }
    \label{fig:minimax_tree_last}
\end{figure}

Teď půjdeme zespodu nahoru.
Uzel porovná hodnoty všech potomků a nastaví svojí hodnotu na tu nejvýhodnější pro hráče na tahu.
Podle toho poté vybereme nejlepší pohyb.\ref{fig:minimax_tree_full}

\begin{figure}
    \centering
    \includegraphics[scale=0.22]{images/minimax_tree_last}
    \caption{Minimax strom. Všechny pozice ohodnocené }
    \label{fig:minimax_tree_full}
\end{figure}

\subsection{Alpha-Beta pruning}\label{subsec:alha-beta-pruning}
Alpha-Beta pruning je vylepšení na algoritmu.
Jde o to, že není třeba vyhodnocovat všechny konečné uzly, protože při postupném vyhodnocování stromu zjistíme, že do některých větví se nikdy nemůžeme dostat za perfektního hraní nepřítele.


\kapitola{Závěr}

\seznamTabulek

\seznamObrazku

\prilohy{
    \kapitola{Příloha}
}


\bibliographystyle{czechiso}
\bibliography{zdroje}

\konec

